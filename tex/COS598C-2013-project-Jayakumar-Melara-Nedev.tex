\documentclass[10pt, conference, compsocconf]{IEEEtran}
\usepackage{amsmath,amssymb}
\usepackage{graphicx}
\usepackage{subfig}
\usepackage{setspace}
\usepackage{multirow}
\usepackage{array}
\usepackage{graphics}
\usepackage{comment}
\usepackage{url}
\usepackage{color}

% correct bad hyphenation here
\hyphenation{}


\begin{document}
%
% paper title
% can use linebreaks \\ within to get better formatting as desired
\title{Towards Minimizing the Required Bandwidth for Mobile Web Browsing}

% author names and affiliations
% use a multiple column layout for up to two different
% affiliations

\author{\IEEEauthorblockN{Madhuvanthi Jayakumar, Marcela Melara, and Nayden Nedev}
\IEEEauthorblockA{Princeton University\\
\{jmadhu, melara, nnedev\}@cs.princeton.edu}}

% make the title area
\maketitle

\begin{abstract}
\end{abstract}

%\begin{IEEEkeywords}
%component; formatting; style; styling;
%\end{IEEEkeywords}

% For peer review papers, you can put extra information on the cover
% page as needed:
% \ifCLASSOPTIONpeerreview
% \begin{center} \bfseries EDICS Category: 3-BBND \end{center}
% \fi
%
% For peerreview papers, this IEEEtran command inserts a page break and
% creates the second title. It will be ignored for other modes.
\IEEEpeerreviewmaketitle

%\category{}{}{}

\section{Introduction}
As of the beginning of 2013, global mobile traffic represented roughly 13\% of internet traffic \cite{?}.
In 2009, this number was just 1\%, moved up to 4\% in 2010, and is expected to grow exponentially over the next X years \cite{?}. Of the 5 billion mobile phones in the world, only a fifth are smartphones \cite{?}. The user base of smartphone is expected to expand by about 42\% a year, and with that grows the mobile web traffic\cite{?}. 
The issue we face is that mobile networks in North America are already running at 80\% of capacity and 36\% of base stations are facing capacity constraints \textbf{[what does ``capacity constraints" mean]} \cite{?}. Globally, the ubiquitous appearance of mobile devices with the rise of cheap smartphones and tablets in developing countries such as Africa and India is creating a demand for available and affortable bandwidth as well. In addition to the computational barriers, the data limits posed on mobile carrier data plans and data overage charges are an incentive for users to utilize their available data effectively. 

%Intro to Previous Work
The growth in demand of mobile network bandwidth in conjunction with the financial incentives of smartphone users motivates new and innovative techniques to reduce mobile network traffic, and to use mobile bandwidth more efficiently. One property of web data pertaining especially to content viewed through a web browser is of particular use for increasing the efficiency of mobile bandwidth utilization: Data redundancy. Specific pages and sites have the tendency to change little over time. Thus, when making a new request for some specific web content, the response will often contain a small number of modifications, but will, for the most part, be identical to a previously requested version of this content. While there has been previous work in this area analyzing redundant desktop browser data \cite{?}, there has been minimal study of data redundancy in the area of mobile browsing.

%Our Method
We present a technique which leverages these data redundancies to improve the bandwith utilization efficiency of mobile browsers. We use data deduplication techniques to find the content a requesting mobile client actually needs, avoiding the transfer of redundant data. By focusing purely on redundancies in web page content, our mechanism helps reduce the number of bytes sent across the network, thereby reducing the required bandwidth. We integrate our technique into the data processing phase of browsing, i.e. the phase between the initial request for a page and the rendering of the requested page. More specifically, we add the following three steps: (1) Data Chunking which partitions incoming web content into fixed-size data chunks, (2) Fingerprinting which uses fingerprinting techniques to create a unique encoding of each unique data chunk, and (3) Caching which adds a layer on top of the web cache to only store unique chunks of data. We use a proxy server to perform the computationally intensive steps (1) and (2) to minimize the additional strain on the limited computational resources of mobile devices.

%Other Methods --> Moved to discussion

%Purpose
Our technique allows us to analyze redundancies across and within websites and to calculate the amount of potential bandwidth savings that can be obtained through data deduplication. As part of our analysis, we test various data chunk and cache sizes with the aim of finding the optimal parameter settings for our technique. We have built a simulator to perform these analyses using use two datasets, one obtained by capturing HTTP response packets through the packet analysis tool Wireshark and the other obtained through telnet requests. We also study various implementations of cache eviction algorithms to determine the optimal choice for our technique. Additionally, we have implemented a proof-of-concept networked mobile client simulator and basic proxy server showing that our technique does not require changes to web server configurations, and does not alter the mobile browsing experience, making this a viable enhancement to mobile browsers benefitting mobile users.






\section{System Design}
\label{sec:sys_design}
Our system is comprised of we add 
the following three steps: (1) Data Chunking which partitions incoming web content into fixed-size data chunks, (2) Fingerprinting which uses fingerprinting 
techniques to create a unique encoding of each unique data chunk, and (3) Caching which adds a layer on top of the web cache to only store unique chunks of data. 
We use a proxy server to perform the computationally intensive steps (1) and (2) to minimize the additional strain on the limited computational resources of mobile 
devices.
\subsection{Data Chunking}
\label{sec:chunking}

\subsection{Data Fingerprinting}
\label{sec:fingerprinting}
Data fingerprinting is the second deduplication technique we use in our bandwidth reduction mechanism. In general, a fingerprinting function is similar to a hash function in that it is also a one-way function which maps an arbitrarily large input to a fixed-size number, and each unique input has a unique fingerprint value. A Rabin fingerprint does this by representing the input data as a polynomial modulo a pre-determined irreducible polynomial \cite{rabin,lbfs}.

Our use of fingerprinting is two-fold: (1) To generate the sliding window chunks, and (2) to generate fingerprints as identifiers for unique chunks. We exploit the fact that any size chunk maps to a fixed-size fingerprint, which is much more efficient to transfer across the network than the entire data chunk. As we discuss in more detail in Section \ref{sec:protocol}, our mechanism requires the transmission of two rounds of fingerprints for data redundancy detection but the total bandwidth savings of our system outweigh these added overhead and bandwidth requirements of these two additional transmissions (see Section \ref{sec:eval}).
\subsection{Caching}
\label{sec:caching}

\section{Reduction Protocol Design}
\label{sec:protocol}



\section{Reduction Protocol Design}
\label{sec:protocol}

\section{Simulator Implementation}
\subsection{Offline Simulator}
\subsection{Networked Simulator}
\subsection{Using Different Caching Algorithms}


\section{Experimental Evaluation}
\label{sec:eval}

We obtained results through a process of collecting offline data, and modifying our simulator to output information about the data being processed. 
Mainly, we observed how changes in parameters affected the miss rate as well as the number of bytes transferred between the proxy and mobile device.

In order to run our experiments, we first collected offline data. 
Over the course of four days, we issued telnet GET requests to various webpages (both desktop and mobile versions) in the morning, afternoon and evening. 
The frequency with which we made these GET requests were for the purpose of reflecting browsing patterns, and it would give us information about the change in the content of a webpage over the course of a day and over the course of multiple days. 
We stored each response in a different file and then processed the data to obtain the byte stream version of the html pages. 
Using this byte stream, we ran several experiments that gave us insight into data redundancy within webpages.

Figure shows the distinctions between mobile web content and desktop web content. 
Many web servers today structure their webpages differently depending on the user-agent they're serving to increase the speed with which the webpages load, to provide better service with respect to UI and various other reasons. 
Therefore, mobile pages are inherently different from desktop browsers and thereby require its own analysis. 
Figure shows that the mobile version of cnn.com is only about a fifth of the size of the desktop version. 
The bytes transferred for the unchunked protocol shows that the size of the webpage remains relatively constant, and that the entire webpage has to be reloaded from the server for each request since the content is no longer "fresh". 
The bytes tranferred with the chunked protocol shows that the amount of redundancy that is eliminated in both mobile and desktop websites is proportional to the size of the web page. 
It also provides insight into exactly where our protocol performs well, and where the overhead of the protocol takes away from the benefits achieved from chunking. 
We see that on the first visit, the amount of bytes that needs to be transferred is almost twice the size of the actual content. 
This inefficiency comes from the fact that we're using chunk size of ten bytes. 
During the first visit to cnn.com, when there is no base copy of the webpage, the fingerprints representing the entire webpage need to be sent back and forth creating an inefficiency. 
However, once there is a base copy in the cache, the overhead decreases substantially. 
We can see from the graph that by the 12th visit, we are only transferring half the number of bytes as we would need to reload the entire webpage. 

The use of chunk size of 10 bytes means that each redundant chunk saves 6 bytes because of the 4 bytes of fingerprint needed to represent that chunk. 
This led us to explore different chunksizes to find the ideal chunk size that takes into consideration the tradeoff between having a low cache miss rate and having a fingerprint map to a bigger chunk.  We obtained the data through visits to cnn once a day for four days. Day 1 is not shown since the cache is empty so the contents of the entire webpage needs to be transferred.
This is innately tied to the size of the content. Figure shows the relationship between \% of web content that is needed (based on cache miss rate) and chunk size based on a series of visits to cnn.com. 
The first visit is not shown since the cache is empty and so 100\% of the content needs to be transferred for all chunk sizes. 
The graph shows data from the second day, assuming cache has already been filled with data from the first visit. 
It is clear from this graph that if we use smaller chunk sizes, the percent of content that needs to be sent decreases. 
The steeper line for chunk 5 when compared to chunk 45 shows that as the number of visits increase, the overlap of smaller chunk sizes increases faster.
However, it means that each fingerprint maps to a smaller chunk and so more fingerprints are needed to represent the small amount of data that needs to be transferred and fewer fingerprints are needed to represent a large amount of data. 
Figure takes into account the effects of the extra bytes that need to be transferred to account for the fingerprints that needs to be transferred to represent redndant chunksIn figure we can see that as the chunk size increased, the miss rate also increased as expected, but the bytes transferred actually decreased. 
This is because if the chunk size is small it gets expensive for the mobile device to communicate which chunks it needs. 
At this point, the ideal chunk size depends on the size of content that needs to be transferred as opposed to percentage.

The next two graphs show what happens when we visisted three websites three times a day for four days to simulate "mobile browsing". 
The data was gathered by visiting cnn.com, nytimes.com and economist.com in an alternating basis three times a day over four days. This graph shows that the if the 'base' content of each webpage is in the cache, then less than 20\% of the content is generally new. The first three requests show that there is some, but not a lot of redundancy between webpages.
Figure shows that on the the first visit, the cache is empty and 100\% of the traffic needs to be transferred. 
For the second website, almost all of it needs to be transferred (~90\%) because of lack of overlap with the first website. 
On the third, the proportion decreases further but a majority of the page still needs to be transferred. 
After this point, we have the base page for all three websites in our cache and only the differences need to be transferred from the proxy, so the proportion of content that needs to be transferred stays below 20\% by the sixth url request. 
This figure calculates the proportion of content that needs to be transferred based on the cache miss rate but does not take into account the additional bytes that have to be transferred due to fingerprints.

Figure shows the total number of bytes that were transferred for the 32 requests. This shows the bandwidth savings obtained from chunking.
We assume that no response is identical to a previous response. 
This means that without chunking, the full webpage has to be reloaded for each request, leading to the linearly increasing number of bytes we see in the graph. 
With chunking however, we see that past the first few requests in which the effects of the overhead are heavy, the number of total bytes transferred rises gradually, and the gap between the bytes transferred grows with the number of requests.


\begin{figure}[h] 
\centering \includegraphics[scale=0.40]{images/desktopmobile.png}
\caption{Desktop vs Mobile Browser page differences.}
\end{figure}

\begin{figure}[h] 
\centering \includegraphics[scale=0.40]{images/chunksize.png}
\caption{Effects of Chunk Size on portion of content needed}
\end{figure}

\begin{figure}[h] 
\centering \includegraphics[scale=0.40]{images/chunksize2.png}
\caption{Effects of chunk size on Bytes Transferred. .}
\end{figure}

\begin{figure}[h] 
\centering \includegraphics[scale=0.40]{images/browsing.png}
\caption{Mobile Web Browsing. }
\end{figure}

\begin{figure}[h] 
\centering \includegraphics[scale=0.40]{images/cumulbrowsing.png}
\caption{Cumulative bytes transferred during browsing. }
\end{figure}

<<<<<<< HEAD


=======
\subsection{Different Cache Eviction Schemes}

\begin{figure}[h]
\centering \includegraphics[scale=0.60]{images/caches.pdf}
\caption{Different Caches Performance}
\end{figure}
>>>>>>> 7e1017e8b29756b31a30a0898241978e50fead8f

\section{Discussion}
\label{sec:discussion}
The process of getting our results provided insight into the redundancies that exist within mobile web pages. We first observed that pages that are designed for mobile browsers are only a fraction of the size of pages designed for desktop browsers. In addition to size, we also know that the way that mobile sites are structured are inherently different through observation. Because of this, we decided that it was worth exploring the types of redundancies that exist within mobile sites. This information can be used in the future to examine how these redundancies are similar to or different from desktop browser site redundancies. It can also be used to design caches that will take advantage of the redundancy patterns that we find. 

As for chunk size, we found that a chunk size close to 10 bytes was optimal in the trade-off between obtaining a low miss-rate and reducing the number of bytes of fingerprints transferred back and forth between the proxy and mobile device. Based on the feedback we received after our presentation, we also implemented the sliding window chunking methodology so that we wouldn't be constrained to a fixed chunk size and so that we would find a larger percent of deduplicated data. 

Lastly, we simulated patterns of mobile web browsing to obtain an approximate result of how much we can expect to reduce bandwidth. We found that if the cache contains an older version of the page, and data from several similar pages, then (with a chunk size of 10 bytes) only about 20\% of the content is new and needs to be retreived from the proxy. 

There were multiple design decisions that we considered before constructing the protocol. We designed our protocol such that fingerprint computation only occurs at the proxy. This is because mobile devices tend to have limited computational capacity and if the heavy lifting is deferred to the proxy, then the mobile device simply has to do look-ups to determine cache hits and misses. However, the drawback in this approach is that we incur the overhead of passing fingerprints across the network, leading to possible latency and additional power consumption. In addition, it also reduces the benefits of bandwidth reductions obtained by chunking.

We build a secondary protocol implementation to address the following few additional concerns. First, There is an inefficiency that comes from the additional communication that occurs between the device and proxy to send fingerprints. Second, using the MRU eviction algorithm for our tests could have possibly restricted by how much we can reduce bandwidth. To consider tradeoffs in communication protocols and eviction schemes, we implemented the protocol with a url-cache as proof of concept and for testing purposes. The higher level approach is shown in Figures 12 and 13. 

\begin{figure}[h] 
\centering \includegraphics[scale=0.40]{images/urlcache-protocol.png}
\caption{Protocol. }
\end{figure} 
\begin{figure}[h] 
\centering \includegraphics[scale=0.40]{images/url-cache-hl.png}
\caption{Url-Cache. }
\end{figure} 

The modified implementation works in the following way. In addition to the previous implementation we added a url-cache, shown in Figure 13, which maps each url request to the set of fingerprints that represent it. If the mobile device contains an older version, it sends the list of fingerprints to the proxy along with the requested url. The proxy then sends back the diff between the mobile's fingerprints and fingerprints of fresh content which it either obtains from its own cache or from the web server. 

Figure 12 shows that it may be possible to reduce latency with this approach due to the decreased amount of communication that needs to occur between the proxy and device. In addition, preliminary testing during implementation showed that in some cases (dependent on the size of the page, whether or not an older version of it is in the cache, and pattern of 'browsing') bandwidth was reduced. The fingerprint overhead remains on the same order because the fingerprint of the entire page still needs to be sent. The bandwidth reduction for some cases came from a lower missrate afforded by the url-cache which was used for the eviction scheme. The Least Recently Added url was evicted from the url cache and the fingerprints that the url mapped to were evicted from the other cache as long as the fingerprint didn't overlap with the other webpages.

We also made several assumptions while implementing the protocol and testing. We assume that each time we made a request that the content in our cache was not fresh, and we did not implement a check for expiration-date or use the 'If-Modified-Since' protocol to check for freshness. We also attempted to collect data from various other websites but ran into several issues. Some websites don't have a different version for mobile devices at all. Some others had a different version, but did not use it as its response based on the User-Agent, and instead specifically required a request of the mobile url. Redirects were also not handled in the way we obtained offline data. Furthermore, we did not check to see if the data was "cacheable" and assumed that all the websites we were working with had cacheable content. Lastly, the results of our tests did not incorporate checks for content overlap between different chunks. We do however, now have an implementation of a sliding window for chunking and fingerprinting in place which is passed along the content stream to catch redundancies between different chunks that would not have been caught with fixed-sized chunks. 

A last, but minor point is the consideration of how integrity of data is affected by possible collision due to fingerprinting. We assume that we do not get collision because the size of our content is small enough that the risk of collision is negligible. 

We have thoroughly tested our implementation in various settings and with a wide range of parameter values. We have evaluated the results of chunking by using data from non-chunked protocol as control. We've done preliminary analysis on using mobile web pages by collecting offline data for desktop browsers and using it as a control. We have experimented with various chunk sizes as well as different caching algorithms. We have evaluated the effectiveness of our protocol both in terms of miss-rate as well as how that translates to the actual number of bytes transferred which gives us better insight into the amount of bandwidth that can be saved. 

\section{Conclusion and Future Work}
\label{sec:conclusion}
We have built a simulator to perform these analyses using use two datasets, one obtained by capturing HTTP response packets through the packet analysis tool Wireshark and the other obtained through telnet requests. 
Additionally, we have implemented a proof-of-concept networked mobile client simulator and basic proxy server showing that our technique does not require changes to web server configurations, and does not alter the mobile browsing experience, making this a viable enhancement to mobile browsers benefitting mobile users.
\cite{manber}


% use section* for acknowledgement
\section*{Acknowledgments}

The authors would like to thank...

\bibliographystyle{plain}
\bibliography{bigdata}

\end{document}
