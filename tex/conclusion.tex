\section{Conclusion and Future Work}
\label{sec:conclusion}
We have built a simulator to perform these analyses using use two datasets, one obtained by capturing HTTP response packets through the packet analysis tool Wireshark and the other obtained through telnet requests. 
Additionally, we have implemented a proof-of-concept networked mobile client simulator and basic proxy server showing that our technique does not require changes to web server configurations, and does not alter the mobile browsing experience, making this a viable enhancement to mobile browsers benefitting mobile users.
\cite{manber}



%Future Work
%- Better caching (take expiration into account in proxy cache, more realistic cache data structure, better implementations of eviction algorithms) --> only difference having expiration makes is in the amount of data that is sent between the proxy server and the actual web server.
%- Networked simulator --> take away headers of packets, use real-time http traffic instead of static packet bytes collected from wireshark, can better simulate multiple mobile devices, and do better latency measurements.
%- Actual implementation of this system: using publicly available server, reconstruct chunked data to actual browser-readable data and/or store it in mobile phone's web cache.
%- Look at mobile app http traffic besides browsing http traffic.
%- not just browser but also APPS!
%- delta encoding
%-future work: effects on latency: getting from proxy will be faster
%Delta encoding of chunks to achieve higher dedup.
%Implementation using smartphones.
%Look at mobile app HTTP traffic.
