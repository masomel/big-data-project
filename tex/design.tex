\section{System Design}
\label{sec:sys_design}
At the core of our technique lies the ad hoc system and protocol we integrate into the existing mobile browsing framework in place today. This system is comprised of a specially configured proxy server, a per-device mobile proxy client, each of which manages specific tasks in our deduplication mechanism. Our protocol specifies the data exchanged between these two components. We integrate our system into the data processing phase of browsing, i.e. the phase between the initial request for a page and the rendering of the requested page.

In our system, the proxy server and mobile proxy clients are logically located between the mobile browser and the Web, managing requests for web pages coming from the browser and engaging in the bandwidth reduction protocol. Most of the computation for our deduplication mechanism occurs on the proxy server so as to minimize the additional strain on the limited computational resources of mobile devices, while the mobile proxy client supplies the proxy server with relevant information it needs to find data redundancies. 

For our deduplication mechnism, we borrow several techniques proven useful for identifying similarities between files and data streams. In particular, we use the following two: 
\begin{enumerate}
\item Chunking: Partition incoming web content into data chunks.
\item Fingerprinting: Create a unique encoding for each unique data chunk.
\end{enumerate}
We explore two chunking techniques and use the implementation for Rabin Fingerprinting given by Broder \cite{broder}. 

In addition, our mechanism requires special caches, one on the proxy server, and one on the mobile proxy client to store unique chunks of data. The client cache replaces the built-in browser web cache. We assume multiple mobile devices connecting to the proxy server and making distinct, un-correlated requests such that our mechanism does not attempt to synchronize the proxy server and client caches.

\subsection{Chunking}
\label{sec:chunking}
\subsubsection{Fixed-Size Chunking}

\subsubsection{Sliding Window Chunking}
The second chunking technique we explore is what we call sliding window chunking, a technique originally presented by Manber in order to find similarities between different files \cite{Manber}. As opposed to partitioning files and streams of data into fixed-size chunks, which are inflexible to file and packet content boundaries, this method allows us to adapt chunk sizes based on the actual contents of the file or packet stream. This is done by choosing a fixed-size window which we slide across the entire contents of a web page, and fingerprinting each region until some number of low-order bits of the fingerprint are all 0. Once this occurs, we have found a breakpoint and the chunk boundary is set to the end of this special window. 

We compute our sliding window chunks based on the equations and parameters presented in \cite{spring} and combine them with the enhancements used in the LBFS content-based breakpoint chunking scheme \cite{lbfs}. Thus, not only do we specify a window size $\beta$ and breakpoint fingerprint value with $\gamma$ zeros in the low-order bits, we also specify a minimum and maximum chunk size. 

The purpose of chosing this scheme over the fixed-size chunking scheme is rather straight-forward. Since the chunks are chosen based on content rather than on position in the web page, minor changes will not affect surrounding chunks. In contrast, any minor change to a web page will probably shift a large number fixed-size chunks by some amount causing changes to a large number of fingerprints, which in turn leads to an overall decrease in redundancy detection. Unlike Manber, and Spring and Wetherall, who use this chunking scheme to fingerprint every possible region in a file or packet and then choose a specific subset of these fingerprints to find redundant data \cite{manber,spring}, we use the same approach as LBFS, i.e. to use these scheme merely for finer-grained chunking of the data and determine redundancies using additional hashing or fingerprinting once the chunks have been found. We discuss Rabin fingerprinting for the purposes of finding redundant data chunks in Section \ref{sec:fingerprinting}. 



\subsection{Data Fingerprinting}
\label{sec:fingerprinting}
Data fingerprinting is the second deduplication technique we use in our bandwidth reduction mechanism. In general, a fingerprinting function is similar to a hash function in that it is also a one-way function which maps an arbitrarily large input to a fixed-size number, and each unique input has a unique fingerprint value. However, unlike an conventional hash function, this fingerprinting function can be decomposed for the incremental computation of a fingerprint. This is done by representing the input as a polynomial modulo a pre-determined irreducible polynomial \cite{rabin,lbfs}.

Our use of fingerprinting is two-fold: (1) To generate the sliding window chunks, and (2) to generate fingerprints as identifiers for unique chunks. We exploit the fact that any size chunk maps to a fixed-size fingerprint, which is much more efficient to transfer across the network than the entire data chunk. As we discuss in more detail in Section \ref{sec:protocol}, our mechanism requires the transmission of two rounds of fingerprints for data redundancy detection but the total bandwidth savings of our system outweigh these added overhead and bandwidth requirements of these two additional transmissions (see Section \ref{sec:eval}).
\subsection{Caching}
\label{sec:caching}
Another main component of our system are two caches - one in the proxy server and one in the 
mobile client. Each of them maintains a mapping between fingerprints and the chunks that they
represent. Each chunk is being fingerprinted and then requested from either of the caches. There
are no specific restrictions to the internal representation of the caches and their size, \textit{i.e.}
how they work internally is irrelevant to the way the other components of the system work.

%\subsection{Bandwidth Reduction Protocol}
\label{sec:protocol}
The second major part of our technique is the reduction protocol between the mobile device and the proxy server. It brings together all the components described in Section \ref{sec:sys_design}. Every time a new page is requested by the mobile browser, the following protocol is performed:
\begin{enumerate}
\item Mobile device sends an HTTP request to the proxy server.
\item Proxy server relays this request to the proper web server.
\item Proxy server performs chunking and fingerprinting of the chunks for all the received web content. It sends all the fingerprints to the mobile device.
\item Mobile device checks its cache for the fingerprints, and creates a list of those it needs. It sends this list to the proxy server.
\item Proxy server creates a list of the needed chunks according to the received needed fingerprints. It sends this list back to the mobile device.
\item Mobile device reconstructs the entire requested page from its cache contents and the received list of needed chunks.
\end{enumerate}






