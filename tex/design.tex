\section{System Design}
\label{sec:sys_design}
At the core of our technique lies the ad hoc system and protocol we integrate into the existing mobile browsing framework in place today. This system is comprised of a specially configured proxy server, a per-device mobile proxy client, each of which manages specific tasks in our deduplication mechanism. Our protocol specifies the data exchanged between these two components. We integrate our system into the data processing phase of browsing, i.e. the phase between the initial request for a page and the rendering of the requested page.

In our system, the proxy server and mobile proxy clients are logically located between the mobile browser and the Web, managing requests for web pages coming from the browser and engaging in the bandwidth reduction protocol. Most of the computation for our deduplication mechanism occurs on the proxy server so as to minimize the additional strain on the limited computational resources of mobile devices, while the mobile proxy client supplies the proxy server with relevant information it needs to find data redundancies. 

For our deduplication mechnism, we borrow several techniques proven useful for identifying similarities between files and data streams. In particular, we use the following two: 
\begin{enumerate}
\item Chunking: Partition incoming web content into data chunks.
\item Fingerprinting: Create a unique encoding for each unique data chunk.
\end{enumerate}
We explore two chunking techniques and use the implementation for Rabin Fingerprinting given by Broder \cite{broder}. 

In addition, our mechanism requires special caches, one on the proxy server, and one on the mobile proxy client to store unique chunks of data. The client cache replaces the built-in browser web cache. We assume multiple mobile devices connecting to the proxy server and making distinct, un-correlated requests such that our mechanism does not attempt to synchronize the proxy server and client caches.

\subsection{Data Chunking}
\label{sec:chunking}

\subsection{Data Fingerprinting}
\label{sec:fingerprinting}
Data fingerprinting is the second deduplication technique we use in our bandwidth reduction mechanism. In general, a fingerprinting function is similar to a hash function in that it is also a one-way function which maps an arbitrarily large input to a fixed-size number, and each unique input has a unique fingerprint value. A Rabin fingerprint does this by representing the input data as a polynomial modulo a pre-determined irreducible polynomial \cite{rabin,lbfs}.

Our use of fingerprinting is two-fold: (1) To generate the sliding window chunks, and (2) to generate fingerprints as identifiers for unique chunks. We exploit the fact that any size chunk maps to a fixed-size fingerprint, which is much more efficient to transfer across the network than the entire data chunk. As we discuss in more detail in Section \ref{sec:protocol}, our mechanism requires the transmission of two rounds of fingerprints for data redundancy detection but the total bandwidth savings of our system outweigh these added overhead and bandwidth requirements of these two additional transmissions (see Section \ref{sec:eval}).
\subsection{Caching}
\label{sec:caching}

%\section{Reduction Protocol Design}
\label{sec:protocol}


