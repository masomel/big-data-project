\subsection{Different Caching Algorithms}
\label{sec:impl_caching}
A substantial number of different web cache eviction schemes have been discussed in the literature~\cite{Wong2006}. Although 
most of our experiments are done with the MRU scheme, we have implemented several others:

\begin{itemize}
\item \textit{Least Recently Used (LRU)}. This replacement policy could be viewed as the opposite of the \textit{MRU} 
one. It removes the item that has been got or put in the earliest moment of time compared to all other items in the 
cache.

\item \textit{Uniformly Random}. This eviction scheme randomly picks an item to be replaced. All the items have equal 
probability to be picked, \textit{i.e.} they are uniformly distributed. It could be considered a scheme where a 
different distribution is used. However, it our case there is no clear distinction between the fingerprints, so we
chose each one to be picked with equal probability.

\item \textit{Least Frequently Used (LFU)}. Here the item that has been got or put the smallest number of times compared
to the all other items in cache is replaced.

\item \textit{LFU with dynamic aging (LFU-DA)}~\cite{Podlipnig2003}. A well-known issue with the \textit{LFU} scheme is
the cache pollution problem. That is, when an item has accumulated a very high frequency count and then becomes unpopular,
it remains for a huge amount of time in the cache before being removed. \textit{LFU-DA} solves this problem by adding a
constant to the frequency count of an item when it is accessed. This way, the recently popular items have a bigger
frequency count that the older ones. It prevents, previously popular items to pollute the cache and accelerates their 
removal.

\end{itemize}
