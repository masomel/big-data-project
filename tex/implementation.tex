\section{Simulator Implementation}
\label{sec:implementation}
We implemented two versions of a simulator of our system: 
\begin{enumerate}
\item An offline simulator, which uses data collected and stored during a mobile browsing session, and input into the simulator offline, and
\item A networked simulator, which simulates a basic incarnation of our system in real-time.
\end{enumerate}
Both simulators are written in Java, and use five helper interfaces and classes each one representing a component of the system, in addition to the proxy server and mobile device classes. In particular, we use two helper interfaces: \texttt{ICache} and \texttt{IProcessor}. \texttt{ICache} allows for different implementations of caches supporting various eviction algorithms. \texttt{IProcessor} allows creating different cache processors. A cache processor is an entity which interfaces a device and its web cache, with its most important task to process incoming web content based on the device's cache contents. While both of our simulators use a single implementation of \texttt{IProcessor} called \texttt{SimpleProcessor}, which manages web content caching, and measures the cache hit-rate and miss-rate, we have multiple implementations of \texttt{ICache}, which we address later in this section. 

The three helper classes we use are \texttt{Chunk}, \texttt{Chunking} and \texttt{Fingerprinting}. The \texttt{Chunk} class defines a chunk with a given size in number of bytes and the data. \texttt{Chunking} is the facility which generates all the data chunks for a given input, either an input file containing web page data or a data stream of online web data. The \texttt{Fingerprinting} class is a wrapper for the Java \emph{rabinhash} library \cite{rabinhash}, and uses 32-bit fingerprints of a given chunk \cite{rabin_api}.

Finally, we created the \texttt{ISimulator} interface to build different kinds of simulators. Our offline version uses one or more \texttt{Mobile} devices and a \texttt{ProxyServer} to implement the simulation of our reduction protocol described in Section \ref{sec:protocol}. The networked simulator uses the networked counterparts of these two components.

Figure \ref{fig:class_diagram} summarizes our implementation software in its entirety\footnote{Blue lines denote aggregation with the cache and are merely for legibility purposes.}.

\begin{figure*}[ht] \centering \includegraphics[scale=0.3]{images/class_diagram.png}
\caption{Implementation Software Class Hierarchy.}
\label{fig:class_diagram}
\end{figure*}

\subsection{Offline Simulator}
\label{sec:offline_sim}
Our implementation of the offline simulator consists of an umbrella simulator (\texttt{SimulatorV1}) which instantiates a proxy server and a mobile device (\texttt{ProxyServer} and \texttt{Mobile}). Since the main goal of this simulator is to analyze mobile web content redundancy, we built it such that it accepts multiple input files containing stored web page data. The chunk size as well as proxy server and mobile cache sizes are parameters to the simulator.

The simulator implements our protocol by passing the appropriate values as variables between the two device instances via specific methods and functions. In order to simulate multiple rounds of our protocol, it then iterates over the specified set of files gathering the following three statistics: (1) The number of chunks (and hence fingerprints) processed, (2) The remaining cache capacity after processing a web page, and (3) The cache miss-rate for the last web page processed. We use these statistics to calculate the average mobile cache miss-rate for one series of protocol simulations, as well as the average number of bytes transmitted between the proxy server and the mobile client.

There is one detail about our offline simulator that is worth noting. After running tests with fix-sized chunks, we began an implementation of this simulator (\texttt{SimulatorV2}) using sliding window chunking in order to measure differences in number of bytes transferred and saved bandwidth. During our preliminary tests we found that our implementation still contains a few bugs and does not accurately perform sliding window chunking for all input web page data. Nevertheless, the results we obtained from the early experiments that came to a successful completion with this simulator, show promising bandwidth reduction rates indicating that this chunking method indeed offers further improvements to our bandwidth reduction system.

\subsection{Networked Simulator}
\label{sec:netsim}
We implemented the networked simulator in its entirety in over 1100 sloc of Java (including all helper classes and interfaces, not including the rabinhash library). The networked simulator consists of the proxy server (\texttt{ProxyServerNet}) and the mobile client simulator (\texttt{SimulatorV3}), which is a wrapper for the networked mobile client (\texttt{MobileClientNet}) and is capable of performing several rounds of requests to the proxy server in real-time. It does so by prompting the user to manually enter the next web page URL she wishes to visit, simulating a web browser (see Figure \ref{fig:mobsim_ui} for an example of the user interface of our mobile client). The simulator architecture can be seen in Figure \ref{fig:netsim_arch}. 

\begin{figure*}[ht] 
\centering \includegraphics[scale=0.40]{images/component_diagram.png}
\caption{Networked Simulator Runtime Interactions.}
\label{fig:netsim_arch}
\end{figure*}

\begin{figure}[h] 
\centering \includegraphics[scale=0.40]{images/mobilesim_ui.png}
\caption{The User Interface of our Mobile Client Simulator.}
\label{fig:mobsim_ui}
\end{figure}

In more detail, the networked simulator implements our reduction protocol as follows. First, the mobile client opens a socket to the proxy server, which is listening for connections at a user-specified port. The mobile client sends a simplified HTTP request of the format \emph{GET $<$url$>$ HTTP/1.1} to the proxy server. It then formulates a proper HTTP request, including the User-Agent string of a mobile device\footnote{We use the Samsung Galaxy SII as our model mobile device across all our implementations and experiments.} to ensure that it receives the mobile version of the requested web page from the hosting web server. Once the proxy has received the content of the requested page from the appropriate web server, it engages in the reduction protocol\footnote{Minor changes were made to the chunking facility as well as the mobile device definition to support networking.} with the mobile client, exchanging the relevant information via the network. At the end, the mobile client reconstructs the content data of the received web page into an HTML file, which can then be viewed in any web browser. After the proxy has served the mobile client's request, the client is able to make a new request and repeat this process.

During each round of the protocol, both the proxy server and the mobile client simulator display three statistics to the user: (1) The number of chunks (and hence fingerprints) processed, (2) The remaining cache capacity after processing a web page, and (3) The cache missrate for the last web page processed. Moving towards our ultimate goal of reducing the required bandwidth for mobile phones to save data plan usage, we can use these statistics to calculate the average mobile cache missrate for one series of protocol simulations, as well as the average number of bytes transmitted between the proxy server and the mobile client. A sample simulator output of these statistics can be seen in Figure \ref{fig:mobsim_output}. We elaborate further on these calculations in Section \ref{sec:eval}.

\begin{figure}[h] 
\centering \includegraphics[scale=0.40]{images/mobilesim_output.png}
\caption{Sample Output of our Mobile Client Simulator.}
\label{fig:mobsim_output}
\end{figure}

We found that, while simulating mobile browsing, for one not using an actual mobile phone, and for another not using a web browser program of any sort, is not realistic, our networked simulator is a good first proof-of-concept prototype showing that our reduction protocol is viable. In our experimental evaluation, we argue that it reduces the required bandwidth compared to the currently required mobile browsing bandwidth.

In the end, we would like to address some caveats of our networked simulator. First, as our mobile client does not have the capabilities of a web browser and it receives preprocessed data from the proxy server, it cannot automatically handle HTTP responses indicating page redirects (i.e. server response code $301$, for example). Thus, our proxy server handles HTTP responses with the server response codes $200$ and $40X$ since the web server returns HTML content with these responses. An additional consequence of the fact that our mobile client is not browser-like is that it must create a local copy of each retrieved web page. Although the content of the web page is reconstructed correctly, since many links within web pages point to relative paths, the retrieved web pages are often rendered incorrectly in the local web browser as it cannot find cascading stylesheets (CSS) and embedded images on the local disk. We propose how to solve these issues in our discussion on future work.

\subsection{Using Different Caching Algorithms}
