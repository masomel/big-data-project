\section{Introduction}
% Motivation
As of the beginning of 2013, global mobile traffic represented roughly 13\% of Internet traffic. This number was just 1\% in 2009 and increased to 4\% in 2010. At present, of the 5 billion mobile phones in the world, only a fifth are smartphones. The user base of smartphones is expected to expand by about 42\% a year, and with that grows the mobile web traffic \cite{olson}. The issue we face in North America is that mobile networks are already running at 80\% of capacity and 36\% of base stations are facing capacity constraints \cite{baldwin}. Globally, the ubiquitous appearance of mobile devices with the rise of cheap smartphones and tablets in developing countries in Africa and India is creating a demand for available and affordable bandwidth as well. In addition to computational barriers, the data limits posed on mobile carrier data plans and data overage charges are an incentive for users to utilize their available data effectively. 

%Intro to Previous Work
The growth in demand of mobile network bandwidth in conjunction with the financial incentives of smartphone users motivates new and innovative techniques to reduce mobile network traffic, and to use mobile bandwidth more efficiently. Prior work has found that one possible solution to the bandwidth constraint problem is to identify and reduce data redundancy in website data \cite{spring}. However, this work has mainly been in the area of desktop browsing, while data redundancy in content viewed through a mobile browsers still remains widely unexplored.  
%%%% Where did you get this from?
%Browsing over mobile wireless networks is still a relatively new area and we currently face four overarching challenges: high cost, high latency, low bandwidth and unreliability. The inherent inefficiencies that cause these issues are communication overhead, redundant transmission and %which protocol are you referring to here?%verbose protocols that make communication through mobile networks expensive in terms of both time and money. 
%%%%%
%We focus on addressing the bandwidth constraint problem by identifying and significantly reducing data redundancy in content viewed through a mobile web browser. 
%Specific pages and sites have the tendency to change little over time. Thus, when making a new request for some specific web content, the response will often contain a small number of modifications, but for the most part will be very similar to a previously requested version of the webpage. 
%While there has been some previous work in analyzing redundant desktop browser data \cite{?}, data redundancy in the area of mobile browsing has been minimally researched. We thus seek to learn how this redundancy compares with mobile browser data. %Do we actually have enough desktop browser data to make this statement?%

% Nayden: I left the old text here before edit it
% Our Method 

%We present a technique which leverages these data redundancies to improve the bandwith utilization efficiency of mobile browsers. We use data deduplication 
%techniques to find the content a requesting mobile client actually needs, avoiding the transfer of redundant data. By focusing purely on redundancies in web 
%page content, our mechanism helps reduce the number of bytes sent across the network, thereby reducing the required bandwidth. We integrate our technique into 
%the data processing phase of browsing, i.e. the phase between the initial request for a page and the rendering of the requested page. 

%Other Methods --> Moved to discussion

% Purpose
%Our technique allows us to analyze redundancies across and within websites and to calculate the amount of potential bandwidth savings that can be obtained through data deduplication. As part of our analysis, we test various data chunk sizes with the aim of finding the optimal parameter settings for our technique. We also study various implementations of cache eviction algorithms to determine the optimal choice for our technique. 

% Marcela: Need to add some discussion about results

%In summary, our main contributions are the following:
%\begin{itemize}
%\item The design of a system and protocol which integrate into the current mobile browsing framework.
%\item The implementation of two simulators, one for experimental purposes, one as a proof-of-concept prototype.
%\item The experimental evaluation of our system and protocol, with which we demonstrate significant improvements to current mobile network bandwidth requirements.
%\end{itemize}

% Nayden: edited version
% Our Method
We present a technique which leverages these data redundancies to improve the bandwidth utilization efficiency of mobile browsers. We use the standard data deduplication 
techniques of chunking \cite{manber,spring,lbfs} and Rabin fingerprinting \cite{rabin} to find the content a requesting mobile client actually needs, avoiding the transfer of redundant data. By focusing purely on redundancies in web 
page content, our mechanism helps reduce the number of bytes sent across the network, thereby reducing the required bandwidth. Our technique allows us to analyze 
the redundancies across and within websites and to calculate the amount of potential bandwidth savings that can be obtained through data deduplication. This paper 
makes the following major contributions:
\begin{itemize}
\item Design of a system and a protocol for effective reduction of required bandwidth for mobile browsing.
%\item Integration of the proposed system and protocol into the current mobile browsing framework. --> This is not true, we have not done this yet. Our system is designed to be integrated but it won't actually be until we do the real implementation of it.
\item Complete implementation of two simulators - one for experimental purposes and another that serves as a proof-of-concept prototype.
\item Thorough experimental evaluation of our prototype system on a large number of different criteria, including various data chunk sizes, different cache eviction scheme
algorighms and transferred data metrics.
\end{itemize}

The rest of this paper is organized as follows. We discuss the previous work done on the problem in Section~\ref{sec:rel_work}. We describe the basic design of the 
system that we built in Section~\ref{sec:sys_design}. Details about our implementation are presented in Section~\ref{sec:implementation} 
and results of experimental evaluation in Section~\ref{sec:eval}. We discuss the results of the evaluation that we performed in Section~\ref{sec:discussion}, give some directions for potential future work and conclude in Section~\ref{sec:conclusion}.
