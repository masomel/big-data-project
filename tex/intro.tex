\section{Introduction}
As of the beginning of 2013, global mobile traffic represented roughly 13\% of internet traffic \cite{?}.
In 2009, this number was just 1\%, moved up to 4\% in 2010, and is expected to grow exponentially over the next X years \cite{?}. Of the 5 billion mobile phones in the world, only a fifth are smartphones \cite{?}. The user base of smartphone is expected to expand by about 42\% a year, and with that grows the mobile web traffic\cite{?}. 
The issue we face is that mobile networks in North America are already running at 80\% of capacity and 36\% of base stations are facing capacity constraints \textbf{[what does ``capacity constraints" mean]} \cite{?}. Globally, the ubiquitous appearance of mobile devices with the rise of cheap smartphones and tablets in developing countries such as Africa and India is creating a demand for available and affortable bandwidth as well. In addition to the computational barriers, the data limits posed on mobile carrier data plans and data overage charges are an incentive for users to utilize their available data effectively. 

%Intro to Previous Work
The growth in demand of mobile network bandwidth in conjunction with the financial incentives of smartphone users motivates new and innovative techniques to reduce mobile network traffic, and to use mobile bandwidth more efficiently. One property of web data pertaining especially to content viewed through a web browser is of particular use for increasing the efficiency of mobile bandwidth utilization: Data redundancy. Specific pages and sites have the tendency to change little over time. Thus, when making a new request for some specific web content, the response will often contain a small number of modifications, but will, for the most part, be identical to a previously requested version of this content. While there has been previous work in this area analyzing redundant desktop browser data \cite{?}, there has been minimal study of data redundancy in the area of mobile browsing.

%Our Method
We present a technique which leverages these data redundancies to improve the bandwith utilization efficiency of mobile browsers. We use data deduplication techniques to find the content a requesting mobile client actually needs, avoiding the transfer of redundant data. By focusing purely on redundancies in web page content, our mechanism helps reduce the number of bytes sent across the network, thereby reducing the required bandwidth. We integrate our technique into the data processing phase of browsing, i.e. the phase between the initial request for a page and the rendering of the requested page. More specifically, we add the following three steps: (1) Data Chunking which partitions incoming web content into fixed-size data chunks, (2) Fingerprinting which uses fingerprinting techniques to create a unique encoding of each unique data chunk, and (3) Caching which adds a layer on top of the web cache to only store unique chunks of data. We use a proxy server to perform the computationally intensive steps (1) and (2) to minimize the additional strain on the limited computational resources of mobile devices.

%Other Methods --> Moved to discussion

%Purpose
Our technique allows us to analyze redundancies across and within websites and to calculate the amount of potential bandwidth savings that can be obtained through data deduplication. As part of our analysis, we test various data chunk and cache sizes with the aim of finding the optimal parameter settings for our technique. We have built a simulator to perform these analyses using use two datasets, one obtained by capturing HTTP response packets through the packet analysis tool Wireshark and the other obtained through telnet requests. We also study various implementations of cache eviction algorithms to determine the optimal choice for our technique. Additionally, we have implemented a proof-of-concept networked mobile client simulator and basic proxy server showing that our technique does not require changes to web server configurations, and does not alter the mobile browsing experience, making this a viable enhancement to mobile browsers benefitting mobile users.





