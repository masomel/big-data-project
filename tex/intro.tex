\section{Introduction}
% Motivation
As of the beginning of 2013, global mobile traffic represented roughly 13\% of internet traffic \cite{?}. In 2009, this number was just 1\%, moved up 
to 4\% in 2010, and is expected to grow exponentially over the next X years \cite{?}. Of the 5 billion mobile phones in the world, only a fifth are 
smartphones \cite{?}. The user base of smartphone is expected to expand by about 42\% a year, and with that grows the mobile web traffic \cite{?}. The 
issue we face is that mobile networks in North America are already running at 80\% of capacity and 36\% of base stations are facing capacity constraints 
\textbf{[what does ``capacity constraints" mean]} \cite{?}. Globally, the ubiquitous appearance of mobile devices with the rise of cheap smartphones and 
tablets in developing countries such as Africa and India is creating a demand for available and affortable bandwidth as well. In addition to the computational 
barriers, the data limits posed on mobile carrier data plans and data overage charges are an incentive for users to utilize their available data effectively. 

%Intro to Previous Work
The growth in demand of mobile network bandwidth in conjunction with the financial incentives of smartphone users motivates new and innovative techniques 
to reduce mobile network traffic, and to use mobile bandwidth more efficiently. One property of web data pertaining especially to content viewed through a 
web browser is of particular use for increasing the efficiency of mobile bandwidth utilization: Data redundancy. Specific pages and sites have the tendency 
to change little over time. Thus, when making a new request for some specific web content, the response will often contain a small number of modifications, 
but will, for the most part, be identical to a previously requested version of this content. While there has been previous work in this area analyzing redundant 
desktop browser data \cite{?}, there has been minimal study of data redundancy in the area of mobile browsing.

% Nayden: More motivation needed here

% Our Method 
We present a technique which leverages these data redundancies to improve the bandwith utilization efficiency of mobile browsers. We use data deduplication 
techniques to find the content a requesting mobile client actually needs, avoiding the transfer of redundant data. By focusing purely on redundancies in web 
page content, our mechanism helps reduce the number of bytes sent across the network, thereby reducing the required bandwidth. We integrate our technique into 
the data processing phase of browsing, i.e. the phase between the initial request for a page and the rendering of the requested page. 

%Other Methods --> Moved to discussion

% Purpose
Our technique allows us to analyze redundancies across and within websites and to calculate the amount of potential bandwidth savings that can be obtained through data deduplication. As part of our analysis, we test various data chunk and cache sizes with the aim of finding the optimal parameter settings for our technique. We also study various implementations of cache eviction algorithms to determine the optimal choice for our technique. 

% Marcela: Need to add some discussion about results

In summary, our main contributions are the following:
\begin{itemize}
\item The design of a system and protocol which integrate into the current mobile browsing framework.
\item The implementation of two simulators, one for experimental purposes, one as a proof-of-concept prototype.
\item The experimental evaluation of our system and protocol, with which we demonstrate significant improvements to current mobile network bandwidth requirements.
\end{itemize}

The rest of this paper is organized as follows. We discuss the previous work done on the problem in Section~\ref{sec:rel_work}. We describe the basic design of the 
system that we built in Section~\ref{sec:sys_design} and ~\ref{sec:protocol}. Details about our implementation are presented in Section~\ref{sec:implementation} 
and results of experimental evaluation in Section~\ref{sec:eval}. We discuss the results of the evaluation that we performed in Section~\ref{sec:discussion}, give 
some directions for potential future work and conclude in Section~\ref{sec:conclusion}.
